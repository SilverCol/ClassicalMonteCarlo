\documentclass[a4paper]{article}
\usepackage[slovene]{babel}
\usepackage[utf8]{inputenc}
\usepackage[T1]{fontenc}
%\usepackage[margin=2cm, bottom=3cm, foot=1.5cm]{geometry}
\usepackage{float}
\usepackage{graphicx}
\usepackage{amsmath}
\usepackage{amssymb}
\usepackage{subcaption}
\usepackage{hyperref}

\newcommand{\tht}{\theta}
\newcommand{\Tht}{\Theta}
\newcommand{\dlt}{\delta}
\newcommand{\eps}{\epsilon}
\newcommand{\thalf}{\frac{3}{2}}
\newcommand{\ddx}[1]{\frac{d^2#1}{dx^2}}
\newcommand{\ddr}[2]{\frac{\partial^2#1}{\partial#2^2}}
\newcommand{\mddr}[3]{\frac{\partial^2#1}{\partial#2\partial#3}}

\newcommand{\der}[2]{\frac{d#1}{d#2}}
\newcommand{\pder}[2]{\frac{\partial#1}{\partial#2}}
\newcommand{\half}{\frac{1}{2}}
\newcommand{\forth}{\frac{1}{4}}
\newcommand{\q}{\underline{q}}
\newcommand{\p}{\underline{p}}
\newcommand{\x}{\underline{x}}
\newcommand{\liu}{\hat{\mathcal{L}}}
\newcommand{\bigO}[1]{\mathcal{O}\left( #1 \right)}
\newcommand{\pauli}{\mathbf{\sigma}}
\newcommand{\bra}[1]{\langle#1|}
\newcommand{\ket}[1]{|#1\rangle}
\newcommand{\id}[1]{\mathbf{1}_{2^{#1}}}
\newcommand{\tinv}{\frac{1}{\tau}}
\newcommand{\s}{\sigma}
\newcommand{\us}{\underline{\s}}

\begin{document}

    \title{\sc\large Višje računske metode\\
		\bigskip
		\bf\Large Klasični Monte-Carlo}
	\author{Mitja Vodnik, 28182041}
	\date{\today}
	\maketitle

    S pomočjo Metropolisovega algoritma bomo raziskali ravnovesna termalna stanja klasičnih Isingovih in Heisenbergovih
    spinov.

    \section{2D Isingov model}

    Vzemimo kvadratno mrežo klasičnih Isingovih spinov:

    \begin{equation}\label{eq1}
        \s_r \in \{ -1, 1 \}, \quad r \in L \subset \mathbb{Z}^2
    \end{equation}

    Energijo take mreže v konfiguraciji spinov $\us$ in magnetnem polju $h$ zapišemo kot:

    \begin{equation}\label{eq2}
        E(\us) = \sum_{<r, r'>} \s_r \s_{r'} - h\sum_r \s_r
    \end{equation}

    Vzamemo še, da so točke v falznem prostoru porazdeljene po Boltzmanovi porazdelitvi:

    \begin{equation}\label{eq3}
        w(\us) = Z^{-1}\exp(-\beta E(\us))
    \end{equation}

    \iffalse
    \begin{figure}
        \centering
        \includegraphics[width = \textwidth]{slika16.pdf}
        \caption{Odvisnost toka $J$ med termostatom in verigo v odvisnosti od časa vzorčenja $\tau$.
        Računano je na anharmonski ($\lambda = 1$) verigi dolžine $N = 10$ in povprečeno po $1000$ vzorcih.
        $J_1$ se nanaša na tok na levi, $J_2$ pa na desni strani verige.}
        \label{slika6}
    \end{figure}
    \fi

\end{document}
