\documentclass[a4paper]{article}
\usepackage[slovene]{babel}
\usepackage[utf8]{inputenc}
\usepackage[T1]{fontenc}
%\usepackage[margin=2cm, bottom=3cm, foot=1.5cm]{geometry}
\usepackage{float}
\usepackage{graphicx}
\usepackage{amsmath}
\usepackage{amssymb}
\usepackage{subcaption}
\usepackage{hyperref}

\newcommand{\tht}{\theta}
\newcommand{\Tht}{\Theta}
\newcommand{\dlt}{\delta}
\newcommand{\eps}{\epsilon}
\newcommand{\thalf}{\frac{3}{2}}
\newcommand{\ddx}[1]{\frac{d^2#1}{dx^2}}
\newcommand{\ddr}[2]{\frac{\partial^2#1}{\partial#2^2}}
\newcommand{\mddr}[3]{\frac{\partial^2#1}{\partial#2\partial#3}}

\newcommand{\der}[2]{\frac{d#1}{d#2}}
\newcommand{\pder}[2]{\frac{\partial#1}{\partial#2}}
\newcommand{\half}{\frac{1}{2}}
\newcommand{\forth}{\frac{1}{4}}
\newcommand{\q}{\underline{q}}
\newcommand{\p}{\underline{p}}
\newcommand{\x}{\underline{x}}
\newcommand{\liu}{\hat{\mathcal{L}}}
\newcommand{\bigO}[1]{\mathcal{O}\left( #1 \right)}
\newcommand{\pauli}{\mathbf{\sigma}}
\newcommand{\bra}[1]{\langle#1|}
\newcommand{\ket}[1]{|#1\rangle}
\newcommand{\id}[1]{\mathbf{1}_{2^{#1}}}
\newcommand{\tinv}{\frac{1}{\tau}}
\newcommand{\s}{\sigma}
\newcommand{\us}{\underline{\s}}
\newcommand{\vs}{\vec{\s}}
\newcommand{\vr}{\vec{r}}
\newcommand{\vq}{\vec{q}}
\newcommand{\vv}{\vec{v}}
\newcommand{\vo}{\vec{\omega}}
\newcommand{\uvs}{\underline{\vs}}
\newcommand{\expected}[1]{\langle #1 \rangle}
\newcommand{\D}{\Delta}

\begin{document}

    \title{\sc\large Višje računske metode\\
		\bigskip
		\bf\Large Klasični Monte-Carlo}
	\author{Mitja Vodnik, 28182041}
	\date{\today}
	\maketitle

    S pomočjo Metropolisovega algoritma bomo raziskali ravnovesna termalna stanja klasičnih Isingovih in Heisenbergovih
    spinov.
    Konkretno si želimo računati pričakovane vrednosti opazljivk $a(\x)$ v visokodimenzionalnem faznem prostoru
    $\x \in \mathcal{R}^N, N \gg 1$, kjer poznamo verjetnostno porazdelitev $w(\x), d\mu(\x) = w(\x)d^N\x$.
    Z Metropolisovim algoritmom pridemo do točk iz faznega prostora $\x_j, j = 1, \ldots, M$ porazdeljenih po
    verjetnostni gostoti $w(\x)$ - pričakovane vrednosti opazljivk tedaj dobimo kot:

    \begin{equation}\label{eq1}
        \expected{a} = \int d^N\x w(\x) a(\x) = \lim_{M \to \infty} \frac{1}{M} \sum_{j=1}^M a(\x_j)
    \end{equation}

    \section{2D Isingov model}

    Vzemimo kvadratno mrežo klasičnih Isingovih spinov:

    \begin{equation}\label{eq2}
        \s_r \in \{ -1, 1 \}, \quad r \in L \subset \mathbb{Z}^2
    \end{equation}

    Energijo take mreže v konfiguraciji spinov $\us$ in magnetnem polju $h$ zapišemo kot:

    \begin{equation}\label{eq3}
        E(\us) = \sum_{<r, r'>} \s_r \s_{r'} - h\sum_r \s_r
    \end{equation}

    Vzamemo še, da so točke v falznem prostoru porazdeljene po Boltzmanovi porazdelitvi:

    \begin{equation}\label{eq4}
        w(\us) = Z^{-1}\exp(-\beta E(\us))
    \end{equation}

    Želimo simulirati obnašanje pričakovanih vrednosti magnetne susceptibilnosti $\chi$ ter toplotne kapacitete $C_V$
    od temperature $T = \frac{1}{\beta}$.
    Ti se izražata kar preko fluktuacij magnetizacije in energije:

    \begin{equation}\label{eq5}
        \chi = \beta (\expected{M^2} - \expected{M}^2), \quad C_V = \beta^2(\expected{E^2} - \expected{E}^2)
    \end{equation}

    Potek simulacije je naslednji: Izberemo si primerno začetno stanje (osnovno, če začnemo pri
    zelo nizki temperaturi oziroma naključno, če začnemo pri zelo visoki), nato se z Metropolisovim algoritmom
    sprehajamo po faznem prostoru in na vsakih nekaj korakov izračunamo opazljivki $M$ in $E$.
    Po končanem sprehodu izračunamo fluktuaciji in tako določimo vrednosti $\chi$ in $C_V$.
    Ta postopek ponavljamo pri različnih vrednostih inverzne temperature $\beta$. \\

    Natančna izvedba Metropolisovega sprehoda je naslednja:

    \begin{enumerate}
        \item Naključno si izberemo mesto $r$ na mreži, kjer bomo poskusili obrniti spin.

        \item Izračunamo razliko v energiji, če bi spin obrnili:
        \begin{equation}\label{eq6}
        \D E_r = 2\s_r \left( \sum_{r'}^{<r, r'>} \s_{r'} + h \right)
        \end{equation}

        \item Če $\D E_r < 0$, spin obrnemo, sicer izberemo enakomerno porazdeljen $\xi \in [0, 1]$ in spin obrnemo če:
        \begin{equation}\label{eq7}
            \xi < \exp(-\beta \D E_r)
        \end{equation}

        \item Če smo spin obrnili, posodobimo vrednosti $M$ in $E$ v sistemu.

    \end{enumerate}

    \subsection{Rezultati}

    Simulacijo poganjamo za $200$ vrednosti inverzne temperature $\beta$ enakomerno razporejenih na intervalu
    $(0, 1]$, pri vsaki temperaturi opravimo po $100 \times N$ Metropolisovih korakov.\\

    Če začnemo pri visoki temperaturi ($\beta = 0$) in sistem ohlajamo, bomo sicer prišli v delno urejeno stanje,
    a nič nam ne garantira, da bomo zadeli osnovnega.
    Primer takega ohlajanja je lepo viden v priloženi animaciji \textit{cool\_ising.gif}.\\

    Najbolj nas zanima prehod med urejenim in neurejenim stanjem, zato je boljše začeti pri nizki temperaturi
    ($\beta = 1$), v osnovnem (popolnoma urejenem) stanju in sistem segrevati.
    Na ta način dobimo sliko \ref{slika1}.
    Na njej je prikazana temperaturna odvisnost magnetizacije $M$, magnetne susceptibilnosti $\chi$ ter toplotne
    kapacitete $C_V$ v okolici faznega prehoda, za različno velike Isingove mreže.

    \begin{figure}
        \centering
        \begin{subfigure}{\textwidth}
            \includegraphics[width = \textwidth]{slika1.pdf}
            \caption{}
        \end{subfigure}
        \begin{subfigure}{\textwidth}
            \includegraphics[width = \textwidth]{slika2.pdf}
            \caption{}
        \end{subfigure}
        \begin{subfigure}{\textwidth}
            \includegraphics[width = \textwidth]{slika3.pdf}
            \caption{}
        \end{subfigure}
        \caption{Fazni prehodi Isingovega modela za različno velike mreže.
        Oranžna črta prilagojena na graf magnetizacije predstavlja Onsagerjevo formulo za spontano magnetizacijo.
        Z rdečo vertikalno črto je označena kritična temperatura določena s to prilagoditvijo.
        Vse je računano pri sklopitveni konstanti $J = 1$.}
        \label{slika1}
    \end{figure}

    Za oceno kritične temperature sistema ($T_c = \frac{1}{\beta_c}$) na graf magnetizacije prilagodimo Onsagerjev
    izraz za temperaturno odvisnost spontane magnetizacije:

    \begin{equation}\label{eq8}
        M(\beta) = \left( 1 - \sinh^{-4} \left( \log(1 + \sqrt{2}) \frac{\beta}{\beta_c} \right) \right)^{\frac{1}{8}}
    \end{equation}

    Vrednost, ki jo ta izraz predvideva (pri sklopitveni konstanti $J = 1$) je:

    \begin{equation}\label{eq9}
        \beta_c = \frac{\log(1 + \sqrt{2})}{2J} \approx 0.44
    \end{equation}

    Mi dobimo podobno, a nekoliko nižjo vrednost.
    Poglejmo si še susceptibilnost $\chi$ ter toplotno kapaciteto $C_V$: obe imata v okolici prehoda maksimum, s tem da
    maksimum prve pada, druge pa raste z večanjem dimenzije mreže.

    \section{1D Heisenbergov model}

    Vzemimo sedaj še enadimenzionalno verigo $N$ Heisenbergovih spinov $|\vs_j| = 1$.
    Energija konfiguracije $\uvs = (\vs_1, \ldots, \vs_N)$ je:

    \begin{equation}\label{eq10}
        E(\uvs) = -\sum_{j=1}^N (J \vs_j \cdot \vs_{j+1} + h\s_j^z)
    \end{equation}

    Simulacija v tem primeru poteka podobno kot v primeru Isingove mreže, spremenijo pa se nekateri detajli
    Metropolisovega sprehoda:

    \begin{enumerate}
        \item Izbira nove orientacije $\vs'_j$ izbranega spina $\vs_j$ je bolj zapletena, saj ima tri dimenzije.
        To je problem naključne izbire točke na sferi $S^2$, postopek je naslednji:
        \begin{enumerate}
            \item Izberemo enakomerno porazdeljena koordinato $z \in [-1, 1]$ in polarni kot $\phi \in [0, 2\pi]$.
            \item Določimo radij preseka sfere na višini $z$: $r = \sqrt{1 - z^2}$
            \item Določimo koordinati $x$ in $y$:
            \begin{equation}\label{eq11}
                x = r\cos{\phi}, \quad y = r\sin{\phi}
            \end{equation}
        \end{enumerate}
        \item Spremembo energije ob zgornji spremembi orientacije spina izračunamo kot:
        \begin{equation}\label{eq12}
            \begin{split}
                \D E_j = &\D \vs'_j \cdot \left( J(\vs_{j-1} + \vs_{j+1}) + h(0, 0, 1) \right), \\
                &\D \vs'_j = \vs_j - \vs'_j = -\D \vs_j
            \end{split}
        \end{equation}
    \end{enumerate}

    Omejimo se le na del faznega prostora, kjer ima veriga ničelno magnetizacijo.
    V tem primeru moramo poskrbeti, da na vsakem koraku Metropolisovega sprehoda ohranimo magnetizacijo - algoritem
    prilagodimo: spreminjamo po dva sosednja spina, tako da se njuna vsota ohranja:

    \begin{equation}\label{eq13}
        \vs'_j + \vs'_{j+1} = \vs_j + \vs_{j+1}
    \end{equation}

    Sprehod je tedaj nslednji:

    \begin{enumerate}
        \item Izberemo nakjučno mesto v verigi $j$.
        \item Določimo os med sosednjima spinoma $\vo = \half (\vs_j + \vs_{j+1})$.
        Os naj določa normalno ravnino $\mathbb{R}^2$ z izhodiščem v koncu osi.
        Nova spina $\vs'_{j}$ in $\vs'_{j+1}$ morata ležati na preseku sfere $S^2$ s to ravnino, to je, na krožnici
        s središčem v $\vo$, ki vsebuje tudi $\vs_j$ in $\vs_{j+1}$.
        \item Če sta spina ravno nasprotna ($|\vo| = 0$), izberemo poljuben $\vs'_j$ enakomerno porazdeljen na sferi,
        vzamemo $\vs'_{j+1} = -\vs'_j$ in preskočimo naslednjih nekaj korakov do računanja energije (točka 7).
        \item Določimo ortogonalno bazo normalne ravnine, ki je normirana na radij preseka:
        \begin{equation}\label{eq14}
            \vr = \vs_{j} - \vo, \quad \vq = \frac{1}{|\vo|} \frac{\vs_{j+1} \times \vs_{j}}{2}
        \end{equation}
        \item Izberemo enakomerno porazdeljen polarni kot $\phi \in [0, 2\pi]$, ter določimo nov spin zapisan v bazi
        normalne ravnine: $\vv = \vr \cos{\phi} + \vq \sin{\phi}$
        \item Predlagana nova spina sta sedaj $\vs'_{j} = \vo + \vv$ in $\vs'_{j+1} = \vo - \vv$
        \item Izračunamo spremembo energije:
        \begin{equation}\label{eq15}
            \D E_j = J(\D \vs'_{j} \cdot (\vs_{j-1} - \vs_{j+2}) + \vs_j \cdot \vs_{j+1} - \vs'_j \cdot \vs'_{j+1})
        \end{equation}
        \item Postopek naprej je že opisan v razdelku o Isingovem modelu.
    \end{enumerate}

    \subsection{Rezultati}

    \ldots

\end{document}
